\chapter{Introduction}

\section{Problématiques}
Les problèmes d'interopérabilité et de réutilisabilité font parties des problématiques majeurs des systèmes informatiques. La programmation objet ne parvient pas à répondre à ces problématiques à cause notamment de ses propriétés de granularité et de fort couplage. Les problèmes de cohésions (pas d'éxplicitaitons des besoins requis) sont un frein à l'utilisation de la programmation objet à grande échelle. 

Parmies les solutions à ces problématiques les architectures à composants sont des alternatives efficaces. Un composant, entité principale, correspond par définition à une ou plusieurs fonctionalité. Il est donc possible de le de le concevoir et de le pré-tester de maanière indépendante. La coopération entres composant est alors plus aisée. De plus un composant possède un couplage lâche. Cette propriété garantissant l'interopératibilté est permise par l'utilisation d'interfaces entre composants. 

\section{Présentation du sujet}
Le module \textsc{Architectures and Components} nous propose une étude de la programmation par composants. L'objectif de ce Projet/TP  est de réaliser un Home Architecture Description Language, et d'implémenter une architecture classique \emph{Clients-Serveur}. La démarche proposée pour atteindre cet objectif est composée de trois étape : 


\begin{enumerate}
\item 
Réalisation du M2 :\hfill \\
Il s'agit de la modélisation d'un méta model.  
\item
Réalisation du M1 :\hfill \\
À partir du méta model précédement réalisé, nous pouvons proposer une instance de ce dernier et obtenir un modèle. Nous choisirons ici de modéliser une application classique \textsc{Client-Serveurs}.  
\item
Réalisation du M0 :\hfill \\
Dernière étape, elle consiste à l'implémentation du M1 et de coder concrétement qui y sont modélisées.
\end{enumerate}

Dans ce rapport nous détaillerons chaque étape la modélisation proposée puis l'implémentation.
