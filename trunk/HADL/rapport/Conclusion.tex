\chapter{Conclusion}\label{chap:COnc}
À ce stade, la modélisation de notre méta-modèle (\cf{} chapitre \ref{chap:M2}) nous a permis de générer un modèle pour notre application \textit{Clients-Serveur} (\cf{} chapitre \ref{chap:M1}). Par la suite il sera nécéssaire de passer à l'étape 3 présentée au chapitre \ref{chap:Intro}. C'est à dire  d'implémenter le méta-modèle M2 en JAVA.

La création de l'application \textit{Clients-Serveur} repose sur l'héritage des classes du méta-modèle. Enfin l'instanciation concrète de ses classes achèvera l'implémentation de nottre application.
