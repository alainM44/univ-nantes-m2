% ICI, les mots clés qu'on va utiliser régulièrement dans le document

% Syntaxe pour créer un nouveau mot-clé : " \newcommand{\kw<nomraccourci>}{<Mot_qui_devra_apparaitre>\xspace} "
% Dans le doc, on utilise ensuite simplement : "\kw<nomraccourci>".

%Exemple avec le mot "tartine" :
\newcommand{\kwtartine}{\textbf{Tartine}\xspace}
\newcommand{\kwtartines}{\textbf{Tartines}\xspace}
\newcommand{\kwobeo}{\textit{Obeo}\xspace}

\newcommand{\kwplay}{\textit{Play!}\xspace}
\newcommand{\kwacceleo}{\textit{Acceleo}\xspace}
\newcommand{\kweclipse}{\textit{Eclipse}\xspace}


\newcommand{\kwsoa}{\textit{SOA}\xspace}
\newcommand{\kwentity}{\textit{Entity}\xspace}
\newcommand{\kwcinematic}{\textit{Cinematic}\xspace}

\newcommand{\kwjava}{\textit{Java}\xspace}
\newcommand{\kwscala}{\textit{Scala}\xspace}

\newcommand{\kwebean}{\textit{Ebean}\xspace}