\section{Le Méta-Modèle Cinématique}
\subsection{Modélisation}
\subsubsection{Vue globale du métamodèle}
Le métamodèle est organisé autour de trois principaux packages:
\begin{itemize}
  \item View: représente les concepts liés à la définition des écrans IHM.
Le package View est construit de la manière suivante:
\newline
-> insérer viewDiagramme
  \item Toolkit: représente les concepts liés à la définition des
  widgets\footnote{Elément visuel d'une interface graphique (bouton, ascenseur,
  liste déroulante, etc.)} IHM.
Le package Toolkit est construit de la manière suivante:
\newline
-> insérer toolkitDiagramme
  \item Flow: permet d'identifier le comportement dynamique des écrans IHM.
  Le flow peut être appréhendé comme une sorte de diagramme d'activités.
Le package Flow est construit de la manière suivante:
\newline
-> insérer flowDiagramme
\end{itemize}

\subsubsection{Le modèle}

****************Description du modèle***************

\subsection{Le générateur}
\subsubsection{Organisation et généricité ?}
\subsubsection{}
\subsection{Résultats obtenus}
