\chapter{Conclusion}\label{chap:COnc}


\section{Travaux effectués (terme pour \guim{travaux terminés} ?)}

Au terme de ce projet, nous proposons un générateur de code déployé sous forme de Plugin Eclipse. À partir des trois modèles (\verb+Shop.entity+, \verb+Shop.soa+, \verb+Shop.cinematique+) l'utilisateur peut, par le biais d'une simple IHM (menu contextuel), lancer la génération d'une application \kwplay prête à l'emploi. Il sera toutefois nécessaire d'ajuster manuellement certaines portions de code, afin d'effectuer le \guim{branchement} des Services entre-eux, ou d'ajouter des niveaux de détail aux différentes opérations.


\section{Problèmes persistants}

En raison d'une mauvaise gestion du temps, tous les aspects n'ont pas pu être abordés, et les générateurs devront être complétés pour faire face à de nouveaux besoins. Nous avons néanmoins structuré les générateurs de façon à garantir leur maintenabilité, en structurant correctement les Templates et Queries et en établissant des conventions de nommage.\\
Il est également à noter que nous avons fait l'impasse sur les tests unitaires, qui peuvent pourtant être considérés comme primordiaux lorsque l'utilisateur/développeur final souhaite modifier le code tout en s'assurant que l'application fonctionne toujours.

\section{Bilan}

Ce Projet de fin d'études nous a permis de découvrir en profondeur les méthodologies du MDA au travers de la solution puissante que représente \kwacceleo. Nous avons également pu expérimenter le Framework \kwplay qui constitue une solution encore jeune mais efficace pour la conception de sites et services Web en Java. La solution que nous avons développée gagnerait à être enrichie et améliorée, mais elle nous a permis de passer par toutes les étapes d'un développement MDA, de la conception des Modèles au système de déploiement de l'application, en passant par la création des générateurs et l'élaboration d'un prototype.
% +++
\\
\\
\kwacceleo{}, et plus généralement les logiciels MDE, constituent des technologies d'avenir, qui permettent à l'utilisateur (ou développeur) final de s'approprier son application et de l'adapter en fonction de ses nouveaux besoins. En effet, une simple modification du Modèle de l'application permet à l'utilisateur de modifier le comportement de son application sans modifier directement le code source, ce qui permet de gagner du temps et de l'argent.
% +++

% Aspect "on a appris des trucs"
% Aspect "on a bossé avec des pro (OBEO, c'est le bien)"
% Aspect "ouverture sur l'intérêt et l'avenir du M2T"


% LocalWords:  méta-modèle Clients-Serveur l'instanciation
% LocalWords:  l'implémentation
