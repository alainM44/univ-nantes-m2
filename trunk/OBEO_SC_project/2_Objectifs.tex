\chapter{Objectifs et démarches}\label{chap:mod}
Dans le cadre de l'élaboration de notre modèle, trois méta-modèles ont été mis à notre disposition par \obeo. Dans ce chapitre, nous détaillerons le rôle de ces trois méta-modèles.


\section{Objectifs}
Dans le cadre du module \emph{SC Project}, nous avons eu l'opportunité de travailler sur la conception d'outils de modélisation dédiés aux applications web. En partant de trois méta models mis à disposition par \textsc{Obeo}
(\cf{} section \ref{sec:met}) l'objectif était de créer des modèles et les générateurs de code associé pour produire le code d'une aplication. L'utilisateur final aura ainsi la possibilité de créer le modèl de son application web (à partir d'une vue en arbre par exemple) puis d'executer le générateur mis à disposition pour produire le code de l'application désirée.





\section{Démarche}
comment on va procéder toussa

