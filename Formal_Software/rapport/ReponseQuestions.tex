\chapter{Réponses aux questions théoriques}\label{chap:Rep}
\section{Modélisation du réseau de Petri}
Nous avons, grâce à \emph{Roméo}, modéliser le système à étudier en un réseau de \textsc{Petri}. Nous avons choisi de représenter le système avec deux files ; dont celle allant de l'automate $A$ vers $B$ étant de taille trois et celle allant de $B$ vers $A$ étant de taille deux.

Une fois le système modélisé, l'étude du graphe de marquages (\emph{c.f.} figure NON GÉNÉRÉE) généré par \emph{Roméo} nous apprend qu'il y a toujours au moins un emplacement de chacune des files qui est vide. On peut donc en conclure que pour notre modèle la file de $A$ vers $B$ est bornée à deux ; tandis que le file de $B$ vers $A$ est bornée à un.

\section{Graphe de marquages}

blabla

\section{Logique temporelle}

Deux propositions nous étaient donnés afin d'être exprimées en logique temporelle:
\begin{enumerate}
 \item Le protocole peut toujours revenir à l’état initial.
 \item Entre deux états où les processus A et B sont en attente, il existe un état où A est déconnecté.
\end{enumerate}
\
Munissons nous de $i$, $i$ faisant partie de notre alphabet pour la logique temporelle et décrivant l'état où A et B sont dans l'état initial. On peut alors exprimer la première proposition de la façon suivante :
\begin{equation}
 \Box\Diamond i
\end{equation}
Pour la seconde proposition, munissons nous de $a$ et $d_A$ appartenant à l'alphabet. $a$ étant l'état où A et B sont en attente et $d_A$ tout état du système où A est déconnecté (dans l'état initial). On peut alors exprimer la seconde proposition de la façon suivante :
\begin{equation}
 \Box((a\wedge\bigcirc\Diamond a)\Longrightarrow(\neg(\neg d_A\cup a)))
\end{equation}
On peut cependant donner une variante où la partie droite de l'implication est modifiée :
\begin{equation}
 \Box((a\wedge\bigcirc\Diamond a)\Longrightarrow\bigcirc(\neg a\cup d_A))
\end{equation}

\section{Étude de la véracité des propositions pour le modèle étudié}
Nous allons à présent déterminer si les deux propositions s'appliquent à notre système. Vous pouvez obtenir une représentation graphique de l'automate du \emph{client} et du \emph{serveur} respectivement sur les figures \ref{fig:ihmCLIENT} et \ref{fig:ihmIA}.

La première proposition est malheureusement ambiguë. En effet il est possible de donner deux sens au verbe pouvoir. Si l'on étudie le système on voit qu'il est toujours possible de trouver une suite d'opérations permettant de revenir à l'état initial. Cependant il est aussi possible d'obtenir une suite d'opérations qui ne revient jamais à l'état initial (Notamment lors d'un cas de boucle suffi \emph{e.g. :} le système peut opérer un échange de données à l'infini).
Le seconde interprétation de la proposition semble cependant plus en adéquation avec une approche concrète, et donne ainsi un contre-exemple. Le système \emph{pourrait} revenir à l'état initial mais cela ne se produira jamais. La première proposition ne semble donc pas valide pour le système donné.

La seconde proposition est quant à elle plus simple à étudiée. Toutes les actions permettant de quitter l'état d'attente (\emph{bgrej} et \emph{bgack}) entraîneront l'automate $A$ soit directement dans l'état déconnecté soit dans un état dans lequel il est nécessaire de passer par la déconnexion de $A$ pour retourner dans l'état d'attente.  La seconde proposition ne semble donc pas valide pour le système donné.




