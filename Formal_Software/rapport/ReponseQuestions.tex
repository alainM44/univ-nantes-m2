\chapter{Réponses aux questions théoriques}\label{chap:Rep}
\section{Modélisation du réseau de Petri}

bla
borné?

\section{Graphe de marquages}

blabla

\section{Logique temporelle}

Deux propositions nous étaient donnés afin d'être exprimées en logique temporelle:
\begin{enumerate}
 \item Le protocole peut toujours revenir à l’état initial.
 \item Entre deux états où les processus A et B sont en attente, il existe un état où A est déconnecté.
\end{enumerate}
Tout d'abord, définissons un ensemble $E$ tous les éléments résultant du produit cartésien de tous les états de A avec ceux de B (soit 16 éléments dans le cas du problème donné).\\
Notre alphabet pour la logique temporelle sera donc tous les éléments de $E$.\\
Munissons nous de $i \in E$, $i$ étant l'état où A et B sont dans l'état initial. On peut alors exprimer la première proposition de la façon suivante :
\begin{equation}
 \Box\Diamond i
\end{equation}
Pour la seconde proposition, munissons nous de $a \in E$ et $d_A \subset E$. $a$ étant l'état où A et B sont en attente et $d_A$ tout état du système où A est déconnecté (dans l'état initial). On peut alors exprimer la seconde proposition de la façon suivante :
\begin{equation}
 \Box((a\wedge\bigcirc\Diamond a)\Longrightarrow(\neg(\neg d_A\cup a)))
\end{equation}



