\chapter{Introduction}\label{chap:Intro}
\section{Présentation du sujet}
L'objectif de ce projet/TP est de réaliser un protocole de transfert de messages entre deux machines avec des listes des canaux de communication \textit{First In First Out}. La démarche proposée par l'énoncé propose les  étapes  suivantes :
  
  
  \begin{enumerate}
  \item 
    Modéliser ce protocole avec l'outil \textsc{Roméo}.
  \item
    Construire le graphe des marquages.
  \item
    Déterminer si les files de communication entre les automates sont bornées.
  \item
    Spécifier en logique temporelle: \hfill \\
    le protocole peut toujours revenir à l’état initial.
  \item 
    Expliciter les propriétés vraies.
  \item
    Programmer une implémentation : \hfill \\
    Une machine est pilotée par l'utilisateur, l'autre par le programme.
  \item
    Tests  :\hfill \\
    rendre observable les différentes transitions.
  \end{enumerate}


% LocalWords:  TP First Roméo implémentation
